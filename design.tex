\documentclass[12pt]{article}
\usepackage{outlines}
\begin{document}
\section{Model}
\subsection{Objects}
\begin{outline}
  \1 Peer
  \2 Each peer maintains the state information for a peer, such as choked or not, any outstanding requests, and so on.
  \1 Piece
  \2 Each piece maintains the state information for a piece.  Its filename, size, completion status, list of blocks, and so on.  They can also write to file when they're complete.
  \1 Block
  \2 Blocks contain their size, start length, and data.
  \1 Torrent
  \2 Main class.  Shared between all threads, the torrent is responsible for passing messages between peers and pieces.
  \1 torrent\_file
  \2 Writes itself to file whenever it's complete.  Keeps track of what pieces belong to it, and its filename.
\end{outline}
\section{design}
The main thread creates a torrent object, which holds the relevant information to communicate between peers, pieces, and files.  It also creates a thread for every peer, which handles the communication and is passed a copy of the torrent.
\section{Sources}
\begin{outline}
\1 http://www.kristenwidman.com/blog/how-to-write-a-bittorrent-client-part-2/
\1 https://wiki.theory.org/BitTorrentSpecification
\1 http://www.ruby-doc.org/stdlib-2.0.0/libdoc/socket/rdoc/index.html
\1 http://zachdex.tumblr.com/post/36792592990/bitttorrent-client-lessons
\end{outline}
\end{document}


